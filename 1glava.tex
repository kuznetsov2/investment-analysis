\section{Теоретические вопросы}
\subsection*{5. Особенности использования прямых иностранных инвестиций в экономике России}
\addcontentsline{toc}{subsection}{5. Особенности использования прямых иностранных инвестиций в экономике России}

В соответствии с российским законодательством под иностранной инвестицией понимается вложение капитала в объект предпринимательской деятельности на территории Российской Федерации в виде объектов гражданских прав, принадлежащих иностранному инвестору, если такие объекты не изъяты из оборота или не ограничены в обороте в соответствии с федеральными законами .

Для России проблема включения в мировое хозяйство вообще и привлечения иностранных инвестиций в экономику, в частности, является очень важной. Однако оценка этой проблемы со стороны как официальных властей, так и общественного мнения неоднозначна.

Официальная точка зрения состоит в признании необходимости привлечения иностранных инвестиций, создания благоприятного инвестиционного климата. В то же время существуют и мнения о нежелательности широкого доступа иностранного капитала в российскую экономику. Крайним выражением подобной точки зрения является тезис об угрозе «распродажи России»международным монополиям.

Другой, более логичной позицией являются взгляды тех предпринимателей, которые в неконтролируемом притоке иностранного капитала видят угрозу серьезной конкуренции для российской промышленности. К тому же они не согласны с низкой ценой предприятий, выставляемых на аукционы в ходе приватизации, в которых участвуют иностранцы. В США и Европе аналогичные объекты стоят гораздо дороже.

И все же объективные законы мировой экономики, опыт международной миграции капитала свидетельствуют о том, что Россия не может стоять в стороне от этого процесса. Становление открытой экономики невозможно без вывоза капитала из России и его импорта в Россию. Однако наша страна обладает определенной спецификой, которая усложняет данный процесс, делает его неповторимым. Это и обширность территории, неравномерность экономического развития отдельных регионов, неразвитость коммуникационной структуры. Это и определенные особенности, связанные с общей отсталостью экономического развития: наличие устаревшего производственного аппарата, чрезмерное развитие военно-промышленного комплекса в ущерб целому ряду гражданских производств, слабое развитие сельского хозяйства и т.п.

В значительной степени готовность инвесторов к вложению капитала в экономику той или иной страны, в том числе и в экономику России, зависит от существующего в ней инвестиционного климата. Это совокупность политических, экономических, юридических, социальных, бытовых и других факторов, предопределяющих в конечном счете степень риска капиталовложений и возможность их эффективного использования.

В настоящее время инвестиционный климат в России в основном является неблагоприятным. Политическая нестабильность, экономический кризис, разгул преступности и другие признаки переходного периода предопределяют крайне низкий рейтинг России у западных организаций, занимающихся сравнительным анализом условий для инвестиций и степени их риска во всех странах мира.

Тем не менее, сам факт присвоения России кредитных рейтингов является положительным. Во-первых, рейтинговое признание расширяет возможности России в привлечении иностранного капитала и во-вторых, свидетельствует о наличии обстоятельств, привлекательных для притока иностранных инвестиций в Россию. К таковым относятся:
\begin{itemize}
	\item богатые природные ресурсы;
\item квалифицированные кадры, способные к быстрому восприятию новейших технологий в производстве и управлений;
\item относительная дешевизна квалифицированной рабочей силы;
\item огромный внутренний рынок;
\item осуществляемый процесс приватизации и возможность участия в нем иностранного капитала.
\end{itemize}

Наличие этих факторов обусловливает то, что иностранные инвестиции в Россию все-таки поступают:
по линии межправительственных соглашений;
\begin{itemize}
	\item от национальных правительственных организаций;
\item из международных финансовых институтов и банков;
\item от различных частных организаций и предпринимателей.
\end{itemize}

Таким образом, иностранный капитал присутствует в России как в государственной (первые две категории), так и в частной (последняя категория) формах, а также как капитал международных организаций неправительственного типа (третья категория).

В качестве государственных инвестиций используются займы, кредиты, техническая помощь. Здесь речь идет об отношениях между государствами, регулируемых международными соглашениями и нормами международного права.

Инвестиции неправительственных международных организаций осуществляются такими организациями, как Мировой банк, Европейский банк реконструкции и развития (ЕБРР) и др. Первыми активными зарубежными участниками инвестиционного процесса в России стали ЕБРР, Мировой банк, а также ряд крупных банков Запада, открывших в нашей стране свои представительства. Комиссия Европейских сообществ приняла специальную программу технической помощи, предусматривающую консультативное содействие и передачу соответствующих технических средств в рамках одобренных проектов. Названные субъекты выступили инициаторами создания особых фондов для деятельности России, ибо практика инвестирования в новые и развивающиеся рынки показала предпочтительность использования коллективных организационных форм (инвестиционных синдикатов, компаний, фондов), благодаря которым снижается риск.

Международный кредит способствует непрерывности производственных процессов, перераспределению капиталов между странами и отраслями производства, перемещению средств в более эффективные и прибыльные сферы экономики, увеличивает размеры накопления капитала и т.д.

В российской практике такая форма инвестирования подпадает под определение «прочие инвестиции», т.е. такие, которые не являются ни прямыми, ни портфельными. К ним относятся:
\begin{itemize}
	\item торговые кредиты, (кредитование экспорта и импорта);
\item кредиты, полученные от международных финансовых организаций (МВФ, Мирового банка, МБРР и т.д.);
\item банковские вклады зарубежных юридических лиц в национальных банках и национальных юридических лиц зарубежных банках;
\item прочие финансовые активы и пассивы, включая просроченные процентные платежи, невыплаченную заработную плату и налоги и т.п.
\end{itemize}

Инвестиции США, накопленные в России на начало 1999г., распределялись следующим образом: прямые инвестиции - 4,2 млрд. долл., или 68\%, портфельные (вложения в российские ценные бумаги) - 0,03 млрд. долл., или 0,5\% и прочие (главным образом различные кредиты) - 1,9 млрд. долл. или 31,5\%.

Частные инвестиции (как прямые, так и портфельные) являются теми, в которых Россия нуждается прежде всего. Одной из наиболее распространенных форм привлечения прямых вложений в российскую экономику является создание предприятий с иностранными инвестициями. Иностранные компании, работающие в России, по типу поведения на нашем рынке можно подразделить на две основные группы. Первая представлена фирмами, сравнительно недавно пришедшими в страну, не ставящими долгосрочных целей по завоеванию рынка, ориентирующимися на получение сверхвысоких доходов в течение короткого времени (путем торговых, посреднических и финансовых операций).

Вторая группа имеет фундаментальный интерес. Она, в отличие от первой, представлена солидными транснациональными компаниями и преследует цели долгосрочного закрепления российском рынке. Пока, однако, подобные фирмы предпочитают скорее обозначить свое присутствие на нашем рынке, нежели масштабно на нем действовать. Такого рода стратегическим инвесторам в условиях рисковой социально- экономической ситуации в России выгоднее ограничиться заключением со своими партнерами лизинговых соглашений о поставке оборудования, а также проведением торговых и контрактных операций.

Все более возрастающую долю в притоке иностранного капитала составляют портфельные инвестиций. Наибольший интерес дм западных покупателей представляют акции предприятий следующих сфер экономики:
\begin{itemize}
	\item топливно-энергетического комплекса (ТЭК);
\item алюминиевой промышленности;
\item связи и коммуникаций;
\item портов и пароходств;
\item производства цемента;
\item производства минеральных удобрений;
\item горнодобывающей промышленности:
\item пищевой промышленности.
\end{itemize}

С точки зрения промышленностей российской экономики, желательно также привлечение иностранных инвестиций в оборонные предприятия, осуществляющие конверсию производства, в мощности по переработке и хранению продовольствия, добыче золота и алмазов и др. Основной поток привлекаемых иностранных инвестиций должен идти на развитие импортозаменяющих отраслей и экспортноориентированных производств.

Тем не менее доля вложений в акции предприятий составляет не менее 10\% от общего объема инвестиций. Большая доля иностранного инвестиционного портфеля (63,5\%) представлена кредитами от международных финансовых организаций, кредитами правительств иностранных государств под гарантии Правительства РФ, прочими кредитами, банковскими вкладами. Это свидетельствует о растущей роли целенаправленной спекулятивной игры на финансовом рынке России, что не способствует развитию реального сектора экономики.

Если говорить о структуре зарубежных инвесторов, то главное место среди них занимает США.

Однако в последнее время увеличивается приток капитала из азиатских «новых индустриальных стран». По мнению ряда специалистов, главные резервы увеличения зарубежных инвестиций в Россию лежат не на Западе, а на Востоке. Страны этого региона отличаются тем, что:
\begin{itemize}
	\item накопили огромный капитал;
\item имеют опыт работы в переходной экономике;
\item меньше, чем западные партнеры, боятся инвестиционных рисков;
\item в большей степени готовы вкладывать капитал в высокотехнологичные объекты и т.д.
\end{itemize}

Хотя российская экономика еще не стала привлекательной для западных инвесторов, ее былая замкнутость основательно подорвана. Несмотря на скромные по международным меркам объемы, иностранные капиталовложения растут, особенно на фоне вялотекущей стабилизации в стране. Даже после августовского кризиса 1998 г. иностранные инвестиции увеличились. В январе  --- июле 1999 г. по сравнению с аналогичным периодом 1998 г. приток иностранных инвестиций в экономику России вырос в 1,6 раза .

Безусловно, все эти средства составляют лишь незначительную долю мировых инвестиций. Уже к концу 1996 г. общий объем накопленных в мире прямых зарубежных инвестиций составлял 2,7 трлн. долл. В 1996г. поток таких инвестиций между странами достиг 315 млрд. долл. 

Правда, определенные надежды можно связывать с новейшей тенденцией  --- переходом западных партнеров к инвестированию и национальной валюте, как принято во всех цивилизованных странах. Причем значительная часть этих рублевых активов формируется за счет реинвестиций прибыли совместных предприятий и полностью подконтрольных компаний, в том числе через взносы в уставный фонд, кредиты зарубежных совладельцев и собственников. По темпам роста они существенно превосходят долларовые вложения.

Тем не менее, несмотря на динамичный характер развития, роль предприятий с участием иностранного капитала в экономике страны невелика. Что касается крупных иностранных инвестиций, то в России их практически нет.

Кроме того, структура привлеченных иностранных активов не способствует освоению промышленного сектора: Предпочтение отдается финансовой сфере. В 1996 г. соотношение поступивших в Россию иностранных инвестиций производственного и непроизводственного назначения значительно изменилось в пользу последних: объемы зарубежных вложений в сферы финансово-кредитной деятельности, страхового и пенсионного обеспечения увеличились в 10 раз .

Серьезные диспропорции обнаруживаются и в распределении иностранных инвестиций по территории страны. Примерно 2/3 сконцентрированы в Центральном экономическом районе, причем в 1996 г. преобладающая часть поступила в Москву  --- в основном в виде кредитов международных финансовых организаций, торговых кредитов, банковских вкладов и прочих поступлений.

Кроме того, при всей внешней широте размаха международных организаций, взявшихся за содействие России в осуществлении рыночных реформ, существенно осязаемых результатов этой активности нет. Общий размер средств, выделенных международным сообществом с 1991 г. по сентябрь 1996 г., составил 79,2 млрд. долл., из которых 52\% уже выплачено. На инвестиции в этой сумме пришлось 28\% (22,2 млрд. долл.), из них 5,3 млрд. уже израсходовано.

Согласно критериям Всемирного банка выделенные инвестиции должны способствовать их притоку и из других источников, причем в десятикратном размере. Пока эти критериальные установки не реализованы, а деятельность МВФ, ЕБРР и др. в основном сводится к обучению российских кадров и осуществлению маломасштабных пилотных проектов. Выяснилось, что аппарат подобных организаций, как правило, тратит на оплату своих сотрудников больше выделяемых по линии технической помощи средств, чем на реальное инвестирование.

Из сказанного ясно: зарубежной капитал на сегодня не внес заметного инвестиционного вклада в обновление экономики России, пока он играет в ней преимущественно демонстрационную роль. В настоящее время Россия является местом борьбы международных компаний за сбыт своих товаров, а не ареной приложения капитала.

С учетом этого некоторые специалисты склоняются к необходимости большего использования различных форм иностранных кредитов, в частности целевых банковских вкладов с покрытием кредита товарными поставками на экспорт (компенсационные соглашения).

Другие считают, что главным источником инвестиционных вливаний в российскую экономику должен стать отечественный частный капитал как находящийся внутри страны, так и эмигрировавший в последние годы за рубеж. По данным Дойче банка (Германия), прямые российские инвестиции в ФРГ в 4--6 раз превышают аналогичные немецкие инвестиции в экономику России . Для ускорения этого процесса предлагается амнистировать владельцев незаконно вывезенного капитала за границу с целью его возврата в Россию.

В порядке решения задач стимулирования иностранных инвестиций в Россию 9 июля 1999 г. принят Закон «Об иностранных инвестициях в Российской Федерации», В законе выделено самое главное - гарантия сохранности и стабильности осуществления инвестиций со стороны государства. В нем закреплены права иностранных инвесторов на инвестиции и полученные от них доходы и прибыли. Помимо этого гарантии прав иностранных инвесторов распространяется на:
\begin{itemize}
	\item возмещение убытков;
\item компенсацию при национализации и реквизиции имущества иностранного инвестора или коммерческой организации с иностранными инвестициями;
\item обеспечение надлежащего разрешения спора в связи с осуществлением инвестиционной и предпринимательской деятельности на территории Российской Федерации иностранным инвестором;
\item использование на территории Российской Федерации и перевод за её пределы доходов, прибыли и других правомерно получаемых денежных сумм;
предоставление иностранному инвестору правового режима, не менее благоприятного, чем национальному инвестору.
\end{itemize}

Для реализации этих прав определен орган, в компетенцию которого входят разработка и осуществление государственной политики в сфере международного инвестиционного сотрудничества. Этим органом является Правительство РФ. Оно выполняет следующие функции:
\begin{itemize}
	\item определяет целесообразность введения запретов и ограничений осуществления иностранных инвестиций на территории Российской Федерации;
\item устанавливает меры по контролю за деятельностью иностранных инвесторов;
\item утверждает перечень приоритетных инвестиционных проектов;
\item разрабатывает и обеспечивает реализацию федеральных программ привлечения иностранных инвестиций;
\item осуществляет контроль за подготовкой и заключением инвестиционных соглашений с иностранными инвесторами о реализации крупномасштабных инвестиционных проектов;
\item осуществляет контроль за подготовкой и заключением международных договоров Российской Федерации о поощрении и взаимной защите инвестиций.
\end{itemize}

Таким образом, главным условием создания благоприятного инвестиционного климата в России остается формирование правовой обстановки, стабильной законодательной базы, обеспечивающей условия максимально эффективной защиты интересов инвесторов.

Продуманная политика привлечения иностранного капитала  --- наиболее эффективный путь выхода из кризиса. Тем не менее, наряду с положительными моментами она имеет ряд недостатков:
\begin{itemize}
	\item приоритетное внимание к отраслям ТЭК приводит не только к ускоренному исчерпанию невозобновляемых ресурсов, но и к дальнейшей гипертрофии добывающих отраслей;
\item слабость государственного регулирования процесса привлечения иностранного капитала имеет негативные последствия для состояния экологии;
\item заниженный курс рубля позволяет западному капиталу скупать важные объекты за бесценок;
\item иностранные капвложения нередко используются как способ отмыва «грязных денег» из стран Запада и т.д.
\end{itemize}

В связи с этим в Законе предусмотрены меры, направленные на борьбу с недобросовестными инвестициями, в частности, указание на необходимость соблюдения иностранными инвесторами антимонопольного законодательства. В качестве некоторых видов нарушений отмечено, например, такое, как создание на территории России коммерческих организаций с иностранными инвестициями или иностранного юридического лица для производства какого-либо пользующегося спросом товара, а затем самоликвидация в целях продвижения на рынок аналогичного товара иностранного происхождения.

Таким образом, государство должно осуществлять весьма гибкую политику в отношении привлечения иностранных инвестиций. С одной стороны, следует формировать благоприятный для их привлечения инвестиционный климат, а с другой   осуществлять строгий контроль за соблюдением национальных интересов \cite{02}.