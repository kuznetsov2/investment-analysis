\subsection*{54. Способы выпуска акций предприятиями, расщепление и консолидация акций}
\addcontentsline{toc}{subsection}{54. Способы выпуска акций предприятиями, расщепление и консолидация акций}

Существуют различные методы привлечения средств инвесторов для организации или расширения деятельности предприятия. В условиях рыночной экономики основным из них является эмиссия долговых и долевых ценных бумаг. Выпуск ценных бумаг в обращение (эмиссия) осуществляется: 
\begin{itemize}
	\item при учреждении акционерного общества и продаже акций его учредителям (владельцам); 
\item при увеличении размеров первоначального уставного капитала путем дополнительного выпуска акций; 
\item при привлечении заемного капитала путем выпуска облигаций хозяйствующего субъекта. 
\end{itemize}

В мировой практике известны различные способы выпуска акций. Коротко охарактеризуем основные из них. 
Наиболее распространенным методом эмиссии является размещение акций через инвестиционные институты, которые покупают весь выпуск и затем продают его по фиксированной цене физическим и юридическим лицам (Offer for Sale Method).Именно этот метод применяется чаще всего, когда происходит приватизация предприятия. 

Следующий способ – продажа непосредственно инвесторам по подписке – отличается от предыдущего тем, что промежуточная продажа всего выпуска акций инвестиционному институту не производится (Offer by Subscription Method или Public Issue by Prospectus Method). Компания, распространяющая свои акции, в этом случае полагается на собственные силы – готовит хороший проспект, проводит широкую рекламу и т.п. Считается, что только процветающие компании с хорошей репутацией могут позволить себе этот метод. 

Еще один распространенный способ – тендерная продажа (Issue by Tender Method).В этом случае один из нескольких инвестиционных институтов покупает у заемщика весь выпуск по фиксированной цене и затем устраивает торг (аукцион), по результатам которого устанавливают оптимальную цену акции. Рассмотрим простейший пример. 

При небольших выпусках акций наиболее популярным является метод размещения акций брокером у небольшого числа клиентов (Placing Method). В этом случае снижаются расходы заемщика по размещению акций (например, нет расходов, связанных с подпиской на акции). Как правило, величина капитала, привлекаемого таким образом, ограничивается; если компания намерена выпустить акций на большую сумму, она должна пользоваться другими методами. 

Согласно российскому законодательству эмиссия ценных бумаг может осуществляться одним из двух способов: 
\begin{itemize}
	\item [а)] частное размещение без публичного объявления и проведения рекламной кампании среди заранее известного ограниченного числа инвесторов (до 100 включительно) во все время обращения данных ценных бумаг или на сумму не более 50млн. руб.; 
	\item [б)] открытая продажа с публикацией и регистрацией проспекта эмиссии среди потенциально неограниченного числа инвесторов или на сумму более 50млн. руб. 
\end{itemize}

между проводимой акционерным обществом дивидендной политикой и рыночной стоимостью ее акций существует тесная взаимосвязь. Однако четкой, формализованной зависимости между размером дивидендных выплат и динамикой курса ценных бумаг не установлено. Это объясняется воздействием на рыночную цену акций большого числа факторов, в том числе тех, которые в принципе не могут быть формализованы (например, психологические, политические).

В экономической литературе и практике разработаны приемы, с помощью которых финансовый менеджер может воздействовать как на рыночную стоимость ценных бумаг компании, так и при определенных условиях на размер будущих дивидендов. Основными такими приемами являются:
\begin{itemize}
	\item дробление акций;
	\item консолидация акций;
	\item выкуп акций.
\end{itemize}

Методика расщепления акций

Дробление акций (или расщепление, Сплит) --- это увеличение количества акций компании посредством уменьшения их номинала. Дробление акций 2 к 1 означает уменьшение их номинала наполовину без изменений в структуре собственного капитала. Обычно к дроблению акций прибегают устойчиво развивающиеся компании, которые стремятся снизить рыночную стоимость своих акций. Многие западные акционерные общества таким способом обеспечивают достаточно высокую ликвидность собственных ценных бумаг и привлекают потенциальных инвесторов.

Решение о проведении дробления акций принимается общим собранием акционеров. В зависимости от рыночной цены акций и преследуемых целей совет директоров компании определяет пропорции дробления. После этого производится замена ценных бумаг: старые акции, имеющиеся у акционеров, заменяются на новые. Принимая решение о дроблении акций, совет директоров должен учитывать необходимость осуществления дополнительных расходов по выпуску новых и изъятию у акционеров старых ценных бумаг.

Прямой зависимости между пропорциями дробления акций и размерами дивидендов не существует. Крайне редко компании в состоянии сохранить уровень дивидендов на одном уровне до и после дробления акций. Как правило, размер дивидендов на новую акцию (меньшим номиналом) ниже, чем на старую. Однако это не означает сокращения дивидендного дохода акционеров. Например, при дроблении акций 2 к 1 и понижении уровня дивидендного выхода на акцию с 2 до 1,5\% акционер, владевший 100 акциями, получит не 200 (100 х 2), а 300 (200  х 1,5) ден. ед. Если величина дивидендов изменяется пропорционально изменению нарицательной стоимости акций, то дробление акций не повлияет на долю каждого акционера в активах компании.

Методика консолидации акций

Консолидация акций (или обратный сплит) --- это изменение номинальной стоимости акций, при котором акционер получает одну новую акцию большего номинала в обмен на определенное количество старых акций. Это один из способов сокращения числа акций компании. Необходимость консолидации акций возникает у компаний при чрезмерном падении рыночной стоимости ценных бумаг. Большинство западных компаний стремится избежать снижения рыночной цены своих акций значительно ниже 10 дол. В случае падения курса акций ниже установленного компанией показателя несколько акций в обращении может быть заменено одной.

Помимо возникновения дополнительных расходов по замещению ценных бумаг на бумаги большего номинала, консолидация акций имеет и другие негативные последствия. Как правило, на рынке объявление компанией о консолидации акций расценивается как сигнал о возникновении у нее финансовых трудностей. И хотя фирма после проведения консолидации акций может сэкономить на затратах по размещению и обслуживанию меньшего количества новых ценных бумаг, негативный эффект от ее проведения смягчается лишь получением впоследствии значительной суммы прибыли.

Уровень дивидендных выплат у компаний после консолидации акций, как правило, невысок, и дивидендный доход на акцию снижается в большей пропорции, чем количество акций в обращении. Следовательно, консолидация акций как методика регулирования курса ценных бумаг должна применяться компаниями крайне осторожно с обязательным учетом возможных негативных последствий.








