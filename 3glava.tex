\subsection*{36. Финансовая устойчивость компании, предприятия}
\addcontentsline{toc}{subsection}{36. Финансовая устойчивость компании, предприятия}

Анализ финансовой устойчивости проводится для того, чтобы определить, насколько предприятие независимо от внешних источников финансирования, с одной стороны, и способно ли оно отвечать по своим долгосрочным обязательствам, обеспечивая в необходимых объемах долгосрочное финансирование своей деятельности, с другой стороны.

В основе оценки финансовой устойчивости лежат коэффициенты, представленные в таблице \ref{fin-ust}


	\small
	\setlength{\extrarowheight}{2pt}
	\begin{longtable}{|l|l|l|l|}
		\caption{Коэффициенты финансовой устойчивости}
		\label{fin-ust}
			\hline
		Показатель & Способ расчета &Нормальное ограничение&Экономический смысл показателя        \\ \hline
		\endhead
		\parbox[c]{3cm}{Коэффициент капитализации (плечо финансового рычага), $\text{У}_1$}& $\text{У}_1 = \dfrac{\text{Заёмный капитал}}{\text{Собственный капитал}}$ & Не выше 1,5 & Показывает, сколько организация привлекла заёмных средств на 1 руб. вложенных в активы собственных средств \\ \hline
		\parbox[c]{3cm}{Собственный капитал в обороте, $\text{У}_2$} & $\text{У}_2 = \text{Капитал и резервы} - \text{Внеоборотные активы}$      & Увеличение показателя является положительной тенденцией & Нулевое или отрицательное значение показателя свидетельствует, что все оборотные (иногда --- и часть внеоборотных) активы сформированы за счет заёмных источников \\ \hline
		\parbox[c]{3cm}{Коэффициент обеспеченности запасов собственными источниками, $\text{У}_3$} & $   \text{У}_3 = \dfrac{(\text{Собственый капитал} - \text{Внеоборотные активы})}{(\text{Запасы} + \text{НДС})}$&   Увеличение показателя рассматривается как положительная тенденция& Показывает достаточность собственных оборотных средств для покрытия запасов \\ \hline
		\parbox[c]{3cm}{Коэффициент автономии (концентрации собственного капитала, независимости), $\text{У}_4$}&$\text{У}_4 = \dfrac{\text{Собственный капитал}}{\text{Валюта баланса}}$  &  $0,4\leq \text{У}_4 \leq 0,6$ & Соизмеряет собственный капитал со всеми источниками финансирования\\ \hline
		\parbox[c]{3cm}{Коэффициент финансирования, $\text{У}_5$}&$\text{У}_5 = \dfrac{\text{Собственный капитал}}{\text{Заемный капитал}}$   & $\text{У}_5 \geq 0,7$     & Показывает, какая часть деятельности финансируется за счет собственных, какая --- за счет заёмных средств     \\ \hline
		\parbox[c]{3cm}{Коэффициент финансовой устойчивости, $\text{У}_6$    }& $ \text{У}_6 = (\text{Собственый капитал} + \text{Долгосрочные обзательства}) : \text{Валюта баланса}     $ &$\text{У}_6 \geq 0,6$       & Показывает, какая часть актива финансируется за счет устойчивых источников       \\ \hline
		\parbox[c]{3cm}{Коэффициент маневренности, $  \text{У}_7 $}&  $ \text{У}_7 = (\text{Собственный капитал} - \text{Внеборотные активы}) : \text{Собственный капитал}$     & Высокие значения $ \text{У}_7 $ положительно характеризуют финансовое состояние. Рекомендованное значение 0,2--0,5     &Показывает, какая часть собственного капитала вложена в оборотные активы, т. е. находится в мобильной форме, позволяющей относительно свободно маневрировать капиталом     \\ \hline 
		\parbox[c]{3cm}{Коэффициент иммобилизации, $ \text{У}_8 $}&   $ \text{У}_8 = \dfrac{\text{Внеоборотные активы}}{\text{Оборотные активы}}     $ &     Данный показатель отражает, как правило, отраслевую специфику фирмы  & Характеризует соотношение постоянных и  текущих активов     \\ \hline
	\end{longtable}




