\subsection*{36. Финансовая устойчивость компании, предприятия}
\addcontentsline{toc}{subsection}{36. Финансовая устойчивость компании, предприятия}

Анализ финансовой устойчивости проводится для того, чтобы определить, насколько предприятие независимо от внешних источников финансирования, с одной стороны, и способно ли оно отвечать по своим долгосрочным обязательствам, обеспечивая в необходимых объемах долгосрочное финансирование своей деятельности, с другой стороны.

В основе оценки финансовой устойчивости лежат коэффициенты, представленные в таблице

\begin{table}[!ht]
	\small
	\caption{My caption}
	\label{my-label}
	\setlength{\extrarowheight}{1mm}
	\begin{tabularx}{\textwidth}{|p{3cm}|K{4cm}|K{3cm}|K{4cm}|}
		\hline
		\multicolumn{1}{|c|}{Показатель}                                   & \multicolumn{1}{c|}{Способ расчета}                                & \multicolumn{1}{c|}{Нормальное ограничение}             & \multicolumn{1}{c|}{Экономический смысл показателя}                                                                                                               \\ \hline
		Коэффициент капитализации (плечо финансового рычага), $Y_1$        & $Y_1 = \dfrac{\text{Заёмный капитал}}{\text{Собственный капитал}}$ & Не выше 1,5                                             & Показывает, сколько организация привлекла заёмных средств на 1 руб. вложенных в активы собственных средств                                                        \\ \hline
		Собственный капитал в обороте, $Y_2$                               & $Y_2 = \text{Капитал и резервы} - \text{Внеоборотные активы}$      & Увеличение показателя является положительной тенденцией & Нулевое или отрицательное значение показателя свидетельствует, что все оборотные (иногда --- и часть внеоборотных) активы сформированы за счет заёмных источников \\ \hline
		Коэффициент обеспеченности запасов собственными источниками, $Y_3$ &                                                                    &                                                         &                                                                                                                                                                   \\ \hline
		&                                                                    &                                                         &                                                                                                                                                                   \\ \hline
		&                                                                    &                                                         &                                                                                                                                                                   \\ \hline
		&                                                                    &                                                         &                                                                                                                                                                   \\ \hline
		&                                                                    &                                                         &                                                                                                                                                                   \\ \hline
		&                                                                    &                                                         &                                                                                                                                                                   \\ \hline
	\end{tabularx}
\end{table}




