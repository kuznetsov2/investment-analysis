\subsection*{25. Формы и методы государственного регулирования инвестиционной деятельности, осуществляемой в форме капитальных вложений}
\addcontentsline{toc}{subsection}{25. Формы и методы государственного регулирования инвестиционной деятельности, осуществляемой в форме капитальных вложений}

Основные принципы государственной инвестиционной политики нашли правовое оформление в «Комплексной программе стимулирования отечественных и иностранных инвестиций в экономику России». Этот документ определял направление деятельности правительства по выходу из инвестиционного кризиса и создание необходимых предпосылок для активизации инвестиционных процессов. В нем была дана характеристика ситуации, сложившейся в тот период в инвестиционной сфере, рассматривались предпосылки роста инвестиций, меры по их стимулированию и основные направления государственной поддержки инвестиционной деятельности, а также выработка организационно-правовых условий инвестирования.
Мероприятия программы были направлены на повышение объема капитальных вложений в экономику России за счет всех источников финансирования, в том числе путем увеличения доли государственных инвестиций. Основная часть государственных инвестиций должна была быть предоставлена частным инвесторам на конкурсной основе. И хотя многие положения программы по причине кризиса не были реализованы, содержащиеся в ней конкретные планы мероприятий по правовому обеспечению эффективного инвестиционного процесса, от основополагающих законодательных актов до инструкций и методических рекомендаций, не утратили своей актуальности.

Также был опубликован Федеральный закон «Об инвестиционной деятельности в Российской Федерации, осуществляемой в форме капитальных вложений». Рядом статей этого Закона определены формы и методы государственного регулирования инвестиционной деятельности, осуществляемой в форме капитальных вложений, которые проводятся органами государственной власти РФ и субъектов федерации.

Государственное регулирование инвестиционной деятельности, осуществляемой в форме капитальных вложений, предусматривает:

\begin{enumerate}
	\item [1)] создание благоприятных условий для развития инвестиционной деятельности путем:

\begin{itemize}
	\item совершенствования системы налогов;
\item применения механизма начисления амортизации и использования амортизационных отчислений;
\item установления субъектам инвестиционной деятельности специальных налоговых режимов, не носящих индивидуального характера;
\item защиты интересов инвестора;
\item предоставления субъектам инвестиционной деятельности льготных условий пользования землей и другими природными ресурсами, не противоречащих законодательству РФ;
\item принятия антимонопольных мер;
\item расширения возможностей использования залогов при осуществлении кредитования;
\item развития финансового лизинга;
\item проведения переоценки основных фондов в соответствии с темпами инфляции;
\end{itemize}
\item [2)] прямое участие государства в инвестиционной деятельности, которое осуществляется в форме капитальных вложений путем: 
\begin{itemize}
	\item разработки, утверждения и финансирования инвестиционных проектов, осуществляемых Россией совместно с иностранными государствами; 
формирования перечня строек и объектов, подлежащих техническому перевооружению для федеральных государственных нужд и финансирования их за счет средств федерального бюджета; 
\item предоставления на конкурсной основе государственных гарантий по инвестиционным проектам за счет средств федерального бюджета и бюджетов субъектов федерации; размещения на конкурсной основе средств федерального бюджета (бюджета развития РФ) и средств бюджетов субъектов Федерации для финансирования инвестиционных проектов; 
\item проведения экспертизы инвестиционных проектов в соответствии с законодательством России; 
\item защиты российских организаций от поставок морально устаревших и материалоемких, энергоемких и ненаукоемких технологий, оборудования, конструкций и материалов; 
\item разработки и утверждения стандартов (норм и правил) и осуществления контроля за их соблюдением; 
\item выпуска облигационных займов, гарантированных целевых займов; 
\item предоставления концессий российским и иностранным инвесторам по итогам торгов (аукционов и конкурсов) в соответствии с законодательством.
\end{itemize}
\end{enumerate}

Значительное место в системе мер государственного регулирования инвестиционной деятельности занимает экспертиза инвестиционных проектов. Все инвестиционные проекты независимо от источников финансирования и форм собственности объектов капитальных вложений до их утверждения подлежат экспертизе в соответствии с законодательством РФ. Экспертиза инвестиционных проектов проводится в целях предотвращения создания объектов, использование которых нарушает права физических и юридических лиц и интересы государства или не отвечает требованиям утвержденных в установленном порядке стандартов (норм и правил), а также для оценки эффективности осуществляемых капитальных вложений.

Строго регламентируя порядок реализации инвестиций, осуществляемых в форме капитальных вложений, государство предоставляет гарантии субъектам инвестиционной деятельности и защищает их капитальные вложения. Так, всем потенциальным инвесторам обеспечиваются равные права при осуществлении инвестиционной деятельности. Капиталовложения могут быть национализированы только при условии предварительного и равноценного возмещения убытков, причиненных субъектам инвестиционной деятельности; они могут быть реквизированы по решению государственных органов в, случаях, порядке и на условиях, которые определены Гражданским кодексом РФ.

В настоящее время утвержден статистический инструментарий по определению эффективности инвестиций в основной капитал по проектам - победителям конкурсов, имеющих государственную поддержку Инструментарий состоит из единовременной формы федерального статистического наблюдения № 1 «Сведения о результатах реализации инвестиционных коммерческих высокоэффективных проектов - победителей конкурсов».

Целью обследования является получение информационной базы для расчетов фактической эффективности и окупаемости по введенным стройкам и объектам.

Система показателей включает следующие характеристики, рассчитываемые по инвестиционным проектам, получившим государственную поддержку:
\begin{itemize}
	\item фактический срок окупаемости инвестиций в основной капитал;
\item фактический бюджетный эффект от реализации инвестиционных проектов;
\item фактический чистый доход;
\item количество созданных рабочих мест;
\item продолжительность строительства (срок реализации инвестиционного проекта).
\end{itemize}
Наряду с Федеральным законом «Об инвестиционной деятельности в Российской Федерации, осуществляемой в форме капитальных вложений» большое значение для привлечения иностранных инвесторов имеет Федеральный закон «О внесении в законодательные акты Российской Федерации изменений и дополнений, вытекающих из Федерального закона «О соглашении о разделе продукции» (№ 32-ФЗ от 10 февраля 1999 г.). Правительством РФ также принят ряд постановлений, направленных на реформирование национальной экономики, выводу ее из кризиса и последующему возрождению на высокотехнологической базе.

В системе мер, предусмотренных Правительством РФ, значительное место занимают мероприятия, направленные на восстановление и развитие инвестиционных процессов. Намечается проведение переговоров с правительствами развитых стран и с крупнейшими прямыми инвесторами о создании на многосторонней или двусторонней основе международного фонда (фондов) поддержки инвестиций в Россию. Весьма характерно и то, что отечественные и иностранные инвесторы будут иметь равные права и обязанности.

Не менее важной задачей Правительство считает реализацию назревших институциональных преобразований. Для чего предполагается объединение «проблемных» организаций в крупные корпорации, в том числе с государственным участием. Наиболее важные задачи таких корпораций состоят в улучшении менеджмента, упорядочении финансовых потоков, перепрофилировании незагруженных мощностей на выпуск высокотехнологичной продукции и т.д. Кроме того, Государственная комиссия по защите прав инвесторов на финансовом и фондовом рынках России должна обеспечивать своевременное рассмотрение апелляций и жалоб отечественных и иностранных инвесторов о нарушении их прав органами государственной власти и принятии неотложных мер.

Предусматривается конкретизировать систему страховых рисков, т.е. обеспечить страховую и гарантийную поддержку российского экспорта и инвестиционных процессов.

Все перечисленные мероприятия, в случае их успешной реализации, должны обеспечить России приток инвестиций и как следствие динамичное развитие экономики \cite{01}.

