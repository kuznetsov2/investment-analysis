\section{Решение задачи (вариант №4)}

% Please add the following required packages to your document preamble:
% \usepackage{multirow}
% \usepackage{graphicx}
\begin{table}[!ht]
	\resizebox{\textwidth}{!}{%
		\begin{tabular}{|l|l|l|l|l|l|l|l|}
			\hline
			\multirow{2}{*}{Проект} & \multirow{2}{*}{IC, руб} & \multicolumn{5}{l|}{Денежные потоки по годам} & \multirow{2}{*}{Ликв. Ст-ть, руб} \\ \cline{3-7}
			&                          & 1       & 2       & 3       & 4      & 5      &                                   \\ \hline
			А                       & -180                     & 110     & 200     & 110     & 160    & 200    & 130                               \\ \hline
			В                       & -200                     & 200     & 140     & 150     & 150    & 250    & 150                               \\ \hline
			С                       & -190                     & 170     & 110     & 140     & 120    & 100    & 120                               \\ \hline
			D                       & -170                     & 150     & 110     & 110     & 170    & 170    & 100                               \\ \hline
		\end{tabular}%
	}
\end{table}

\textit{Ставку дисконтирования выбрать и обосновать.
	Найти: Простую норму прибыли, коэф-т эффект-ти инвестиций, срок окупаемости, дисконтир-й срок окупаемости, чистый дисконтированный доход, внутреннюю норму доходности ($ROI$, $ARR$, $PP$, $DPP$, $NPV$, $PI$, $IRR$).}

\begin{center}
	Решение
\end{center}

Из теории инвестиционного анализа предполагается, что ставка дисконтирования должна включать минимально гарантированный уровень доходности (не зависящий от вида инвестиционных вложений), темп инфляции, и коэффициент, учитывающий степень риска конкретного инвестирования. То есть этот показатель отражает минимально допустимую отдачу на вложенный капитал (при которой инвестор предпочтет участие в проекте альтернативному вложению тех же средств в другой проект с сопоставимой степенью риска).

Рассчитаем ставку дисконтирования методом кумулятивного построения:
\[ r = r_b + \sum\limits_{i=1}^{n} R_i, \]
где  $r$ --- ставка дисконта,
$r_b$ --- базовая (безрисковая) ставка,
$R_i$ --- премия за $i$-вид риска,
$i$ --- количество премий за риск.

В качестве безрисковой ставки доходности примем размер ключевой ставки ЦБ --- 7,75\%. 

Инфляция на январь 2019 года, по данным ЦБ, составила 5\%. Прогноз по инфляции --- 4\%.

Премию за риск рассчитывается с учетом отраслевой принадлежности предприятия, ситуации на рынке. Примем это значение в размере 5\%.

\[ r = 7,75  + 4 + 5 = 16,75 \%\]

Полученные на этапах расчёта показатели будем вносить в таблицу \ref{final11}.

1. Рассчитаем простую норму прибыли ($ROI$):

Простую норму прибыли определим по формуле \cite[174]{sergeev}:
\[ ROI = \dfrac{\text{Величина годовой чистой прибыли}}{\text{Общая величина инвестиционных затрат}}\]

Так как прибыль по годам реализации проекта распределена не равномерно, в расчете используем среднее значение прибыли за 5 лет.

\[ ROI_A = \dfrac{156}{180} = 0,87\]

\[ ROI_B = \dfrac{178}{200}=0,89\]

\[ ROI_C = \dfrac{128}{190}=0,67\]

\[ ROI_D =  \dfrac{142}{170}=0,84\]

2. Рассчитаем коэффициент эффективности инвестиций ($ARR$):

Наиболее распространенный алгоритм расчета коэффициента эффективности инвестиций ($ARR$) следующий \cite[214]{leontev}:

\[ ARR = \dfrac{\text{Среднегодовая чистая прибыль}}{0,5 \times (\text{Средняя величина инв-ий} + \text{Остаточная ст-ть основного к-ла})} \times 100 \]

\[ ARR_A =  \dfrac{156}{0,5 \times (180 + 130)} \times 100 =100,6\]

\[ ARR_B =   \dfrac{178}{0,5 \times (200 + 150)} \times 100 =101,7\]

\[ ARR_C =  \dfrac{128}{0,5 \times (190 + 120)} \times 100 =82,6\]

\[ ARR_D =  \dfrac{142}{0,5 \times (170 + 100)} \times 100 =105,2\]

3. Рассчитаем простой срок окупаемости ($PP$):

Целую часть периода окупаемости найдем из таблицы \ref{nakopl}.

% Please add the following required packages to your document preamble:
% \usepackage{graphicx}
\begin{table}[!h]
	\label{nakopl}
	\caption{Сальдо накопленного денежного потока}
	\resizebox{\textwidth}{!}{%
		\begin{tabular}{|l|l|l|l|l|l|l|l|}
			\hline
			Проект & IC, руб & 1   & 2   & 3   & 4   & 5   & Ликв. Ст-ть, руб \\ \hline
			А      & 180     & 110 & 200 & 110 & 160 & 200 & 130              \\ \hline
			В      & 200     & 200 & 140 & 150 & 150 & 250 & 150              \\ \hline
			С      & 190     & 170 & 110 & 140 & 120 & 100 & 120              \\ \hline
			D      & 170     & 150 & 110 & 110 & 170 & 170 & 100              \\ \hline
			А      & -180    & -70 & 130 & 240 & 400 & 600 & 130              \\ \hline
			В      & -200    & 0   & 140 & 290 & 440 & 690 & 150              \\ \hline
			С      & -190    & -20 & 90  & 230 & 350 & 450 & 120              \\ \hline
			D      & -170    & -20 & 90  & 200 & 370 & 540 & 100              \\ \hline
		\end{tabular}%
	}
\end{table}

Рассчитаем $PP$ \cite[109]{kasyanenko2}:

\[ PP_A =  1 + \dfrac{|-70|}{|-70| + 130} = 1.35\]

\[ PP_B =  1 + \dfrac{|0|}{|0| + 140} = 1\]

\[ PP_C =  1 + \dfrac{|-20|}{|-20| + 90} = 1,18 \]

\[ PP_D = 1+ \dfrac{|-20|}{|-20| + 90} = 1,18 \]

4. Рассчитаем дисконтированный срок окупаемости ($DPP$):

Продисконтируем денежные потоки по ставке дисконта 16,75\%. Данные внесем в таблицу \ref{diskont}.
% Please add the following required packages to your document preamble:
% \usepackage{graphicx}
\begin{table}[!h]
	\label{diskont}
	\caption{Дисконтированные денежные потоки}
	\resizebox{\textwidth}{!}{%
		\begin{tabular}{|l|l|l|l|l|l|l|l|}
			\hline
			Проект          & IC, руб & 1     & 2     & 3     & 4     & 5     & Ликв. Ст-ть, руб \\ \hline
			А               & -180     & 110   & 200   & 110   & 160   & 200   & 130              \\ \hline
			В               & -200     & 200   & 140   & 150   & 150   & 250   & 150              \\ \hline
			С               & -190     & 170   & 110   & 140   & 120   & 100   & 120              \\ \hline
			D               & -170     & 150   & 110   & 110   & 170   & 170   & 100              \\ \hline
			Дисконт-й мн-ль &         & 0,857 & 0,734 & 0,628 & 0,538 & 0,461 &                  \\ \hline
			А               &         & 94,2  & 171,3 & 94,2  & 137,0 & 171,3 &                  \\ \hline
			В               &         & 171,3 & 119,9 & 128,5 & 128,5 & 214,1 &                  \\ \hline
			С               &         & 145,6 & 94,2  & 119,9 & 102,8 & 85,7  &                  \\ \hline
			D               &         & 128,5 & 94,2  & 94,2  & 145,6 & 145,6 &                  \\ \hline
		\end{tabular}%
	}
\end{table}

Рассчитаем дисконтированный срок окупаемости ($DPP$):
% Please add the following required packages to your document preamble:
% \usepackage{graphicx}
\begin{table}[]
	\caption{Накопленное дисконтированное сальдо}
	\resizebox{\textwidth}{!}{%
		\begin{tabular}{|l|l|l|l|l|l|l|}
			\hline
			Проект & IC, руб & 1     & 2    & 3     & 4     & 5     \\ \hline
			А      & -180    & -85,8 & 85,5 & 179,7 & 316,8 & 488,1 \\ \hline
			В      & -200    & -28,7 & 91,2 & 219,7 & 348,2 & 562,3 \\ \hline
			С      & -190    & -44,4 & 49,8 & 169,7 & 272,5 & 358,2 \\ \hline
			D      & -170    & -41,5 & 52,7 & 146,9 & 292,5 & 438,1 \\ \hline
		\end{tabular}%
	}
\end{table}

\[ DPP_A =  1 + \dfrac{|-85,8|}{|-85,8| + 85,5} = 1,5\]

\[ DPP_B =  1 + \dfrac{|-28,7|}{|-28,7| + 91,2} = 1,24\]

\[ DPP_C =  1 + \dfrac{|-44,4|}{|-44,4| + 49,8} = 1,47 \]

\[ DPP_D = 1+ \dfrac{|-41,5|}{|-41,5| + 52,7} = 1,44 \]

5. Рассчитаем чистый дисконтированный доход ($NPV$):

\[ NPV_A =  -180 + 94,2  +171,3 + 94,2  + 137,0 + 171,3  = 488\]

\[ NPV_B =-200 +171,3 + 119,9 + 128,5 + 128,5 + 214,1 =562,3\]

\[ NPV_C =  -190+145,6+94,2+119,9+102,8+85,7=358,2\]

\[ NPV_D = -170+128,5+94,2+94,2+145,6+145,6=438,1\]

6. Рассчитаем индекс рентабельности (прибыльности) инвестиций ($PI$):

Индекс рентабельности рассчитаем по формуле \cite[218]{leontev}:

\[ PI = \dfrac{\text{Настоящая стоимость денежных поступлений}}{\text{Сумма инвестиций}} \]

\[ PI_A = \dfrac{94,2 +	171,3 +	94,2 	+137,0 +171,3
}{180}= 3,71\]

\[ PI_B = \dfrac{171,3 +	119,9 +	128,5 	+128,5 
+}{200}=3,81 \]

\[ PI_C = \dfrac{145,6 +	94,2 +	119,9 +	102,8 +}{190}= 2,89\]

\[ PI_D = \dfrac{128,5 +	94,2 +	94,2 +	145,6 
}{170}= 3,58\]

7. Рассчитаем внутреннюю норму доходности ($IRR$):







% Please add the following required packages to your document preamble:
% \usepackage{graphicx}
\begin{table}[!h]
		\label{final11}
	\caption{Показатели эффективности инвестиционных проектов}
		\resizebox{\textwidth}{!}{%
			\begin{tabular}{|l|l|l|l|l|l|l|l|}
			\hline
			Проект & ROI  & ARR    & PP       & DPP & NPV & PI & IRR \\ \hline
			А             & 0,87 & 100,6   & 1,35     & 1,5    &  488   &3,71    &     \\ \hline
			В             & 0,89 & 101,7   &  1         &    1,24 &  562,3   &  3,81  &     \\ \hline
			С             & 0,67 & 82,6     &    1,18  & 1,47    &  358,2   &  2,89  &     \\ \hline
			D            & 0,84 & 105,2    &  1,18    &  1,44   &  438,1   & 3,58   &     \\ \hline
		\end{tabular}%
	}
\end{table}

