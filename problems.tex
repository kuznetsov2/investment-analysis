\section{Решение задачи (вариант №4)}

% Please add the following required packages to your document preamble:
% \usepackage{multirow}
% \usepackage{graphicx}
\begin{table}[!ht]
	\resizebox{\textwidth}{!}{%
		\begin{tabular}{|l|l|l|l|l|l|l|l|}
			\hline
			\multirow{2}{*}{Проект} & \multirow{2}{*}{IC, руб} & \multicolumn{5}{l|}{Денежные потоки по годам} & \multirow{2}{*}{Ликв. Ст-ть, руб} \\ \cline{3-7}
			&                          & 1       & 2       & 3       & 4      & 5      &                                   \\ \hline
			А                       & -180                     & 110     & 200     & 110     & 160    & 200    & 130                               \\ \hline
			В                       & -200                     & 200     & 140     & 150     & 150    & 250    & 150                               \\ \hline
			С                       & -190                     & 170     & 110     & 140     & 120    & 100    & 120                               \\ \hline
			D                       & -170                     & 150     & 110     & 110     & 170    & 170    & 100                               \\ \hline
		\end{tabular}%
	}
\end{table}

\textit{Ставку дисконтирования выбрать и обосновать.
	Найти: Простую норму прибыли, коэф-т эффект-ти инвестиций, срок окупаемости, дисконтир-й срок окупаемости, чистый дисконтированный доход, внутреннюю норму доходности ($ROI$, $ARR$, $PP$, $DPP$, $NPV$, $PI$, $IRR$).}

\begin{center}
	Решение
\end{center}

Из теории инвестиционного анализа предполагается, что ставка дисконтирования должна включать минимально гарантированный уровень доходности (не зависящий от вида инвестиционных вложений), темп инфляции, и коэффициент, учитывающий степень риска конкретного инвестирования. То есть этот показатель отражает минимально допустимую отдачу на вложенный капитал (при которой инвестор предпочтет участие в проекте альтернативному вложению тех же средств в другой проект с сопоставимой степенью риска).

Рассчитаем ставку дисконтирования методом кумулятивного построения:
\[ r = r_b + \sum\limits_{i=1}^{n} R_i, \]
где  $r$ --- ставка дисконта,
$r_b$ --- базовая (безрисковая) ставка,
$R_i$ --- премия за $i$-вид риска,
$i$ --- количество премий за риск.

В качестве безрисковой ставки доходности примем размер ключевой ставки ЦБ --- 7,75\%. 

Инфляция на январь 2019 года, по данным ЦБ, составила 5\%. Прогноз по инфляции --- 4\%.

Премию за риск рассчитывается с учетом отраслевой принадлежности предприятия, ситуации на рынке. Примем это значение в размере 5\%.

\[ r = 7,75  + 4 + 5 = 16,75 \%\]

1. Рассчитаем простую норму прибыли ($ROI$):

\[ ROI = \dfrac{\text{Денежный поток} - \text{Объем инвестиций}}{\text{Объем инвестиций}}\]

\[ ROI_B = \dfrac{числитель}{знаменатель}\]

\[ ROI_C = \dfrac{числитель}{знаменатель}\]

\[ ROI_D =  \dfrac{числитель}{знаменатель}\]



